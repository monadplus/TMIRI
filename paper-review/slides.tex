\documentclass[xcolor=pdftex,dvipsnames,table]{beamer}

\mode<presentation>
{
  \usetheme{Singapore}
  \usecolortheme{seahorse}
  \setbeamercovered{dynamic} %invisible
}

\usepackage[utf8]{inputenc}
\usepackage[english]{babel}
\usepackage{minted}
\usepackage[percent]{overpic}

\usepackage{tikz}
% \def\checkmark{\tikz\fill[scale=0.4](0,.35) -- (.25,0) -- (1,.7) -- (.25,.15) -- cycle;}

\graphicspath{ {./images/} }

\def\checkmark{\includegraphics[height=0.5cm]{checkmark.png}}
\def\crossmark{\includegraphics[height=0.5cm]{crossmark.png}}
\def\questionmark{\includegraphics[height=0.5cm]{question_mark.png}}

\definecolor{bgConsole}{HTML}{ffffcc}

%%%%%%%%%%%%%%%%%%%%%%%%%%%%%%

\title{Effect Handlers in Scope}
% \subtitle{}
\author[Wu, Schrijvers, Hinze]
{N.~Wu\inst{1} \and T.~Schrijvers\inst{2} \and R.~Hinze\inst{1}}

\institute[VFU] % (optional)
{
  \inst{1}%
  University of Oxford
  \and
  \inst{2}%
  Ghent University
}

\date[ICFP 2014]{ICFP 2014}

% \logo{\includegraphics[height=0.5cm]{haskell-logo.png}}

%%%%%%%%%%%%%%%%%%%%%%%%%%%%%%

\AtBeginSection[]
{
  \begin{frame}<beamer>
      \frametitle{Table of Contents}
      \tableofcontents[currentsection]
  \end{frame}
}

\begin{document}

\frame{\titlepage}

\begin{frame}
  \begin{columns}[c]
    \column{0.4\textwidth}
    \begin{center}
      \Large{\textcolor{RoyalBlue}{Monad Transformers}}\\
      \small{Traditional Approach \\
      (Liang et al. 1995)}
    \end{center}

    \column{0.4\textwidth}
    \begin{center}
      \Large{\textcolor{ForestGreen}{Algebraic Effect Handlers}}\\
      \small{Recent Developments} \\
      \small{(Plotkin\&Power 2002, \\
      Kiselyov et al. 2013, \\
      Kammar et al. 2013, \\
      Brady 2013)}
    \end{center}
  \end{columns}
\end{frame}

\begin{frame}
  \begin{columns}[c]
    \column{0.3\textwidth}
    \begin{center}
      \Large{\textcolor{RoyalBlue}{Monad Transformers}}
    \end{center}
    \column{0.3\textwidth}
    \column{0.3\textwidth}
    \begin{center}
      \Large{\textcolor{ForestGreen}{Algebraic Effect Handlers}}
    \end{center}
  \end{columns}
  \bigskip
  \begin{columns}[c]
    \column{0.3\textwidth}
    \begin{center}
      \crossmark
    \end{center}
    \column{0.3\textwidth}
    \begin{center}
      \textbf{Methodology}
    \end{center}
    \column{0.3\textwidth}
    \begin{center}
      \checkmark
    \end{center}
  \end{columns}
  \begin{columns}[c]
    \column{0.3\textwidth}
    \begin{center}
      \crossmark
    \end{center}
    \column{0.3\textwidth}
    \begin{center}
      \textbf{Composition}
    \end{center}
    \column{0.3\textwidth}
    \begin{center}
      \checkmark
    \end{center}
  \end{columns}
\end{frame}

\begin{frame}
  \begin{columns}[c]
    \column{0.3\textwidth}
    \begin{center}
      \Large{\textcolor{RoyalBlue}{Monad Transformers}}
    \end{center}
    \column{0.3\textwidth}
    \column{0.3\textwidth}
    \begin{center}
      \Large{\textcolor{ForestGreen}{Algebraic Effect Handlers}}
    \end{center}
  \end{columns}
  \bigskip
  \bigskip
  \begin{columns}[c]
    \column{0.3\textwidth}
    \begin{center}
      \questionmark
    \end{center}
    \column{0.3\textwidth}
    \begin{center}
      \textbf{Effect Interaction}
    \end{center}
    \column{0.3\textwidth}
    \begin{center}
      \questionmark
    \end{center}
  \end{columns}
\end{frame}

{
\setbeamercolor{frametitle}{fg=white, bg=RoyalBlue}
\begin{frame}[fragile, t]
  \frametitle{Monad Transformers}
  \begin{center}
    \textbf{\Large{Effect Interaction?}}
  \end{center}
  \begin{minted}[bgcolor=bg, fontsize=\footnotesize]{haskell}
decr :: (MonadState Int m, MonadExcept () m)
     => m ()
decr = do x <- get
          if x > 0 then put (pred x)
                   else throw ()
  \end{minted}
  \pause
  \begin{minted}[bgcolor=bgConsole, fontsize=\footnotesize]{haskell}
ghci> (runId . runStateT 0 . runExceptT) decr
(Left (), 0)
  \end{minted}
  \pause
  \begin{minted}[bgcolor=bgConsole, fontsize=\footnotesize]{haskell}
ghci> (runId . runExceptT . runStateT 0) decr
Left ()
  \end{minted}
\end{frame}
}

\begin{frame}
  \begin{columns}[c]
    \column{0.3\textwidth}
    \begin{center}
      \Large{\textcolor{RoyalBlue}{Monad Transformers}}
    \end{center}
    \column{0.3\textwidth}
    \column{0.3\textwidth}
    \begin{center}
      \Large{\textcolor{ForestGreen}{Algebraic Effect Handlers}}
    \end{center}
  \end{columns}
  \bigskip
  \bigskip
  \begin{columns}[c]
    \column{0.3\textwidth}
    \begin{center}
      \checkmark
    \end{center}
    \column{0.3\textwidth}
    \begin{center}
      \textbf{Effect Interaction}
    \end{center}
    \column{0.3\textwidth}
    \begin{center}
      \questionmark
    \end{center}
  \end{columns}
\end{frame}

{
\setbeamercolor{frametitle}{fg=white, bg=ForestGreen}
\begin{frame}[fragile, t]
  \frametitle{Algebraic Effect Handlers}
  \begin{center}
    \textbf{\Large{Effect Interaction?}}
  \end{center}
  \begin{minted}[bgcolor=bg, fontsize=\footnotesize]{haskell}
decr :: (State Int <: sig, Exc () <: sig)
     => Prog sig ()
decr = do x <- get
          if x > 0 then put (pred x)
                    else throw ()
  \end{minted}
  \pause
  \begin{minted}[bgcolor=bgConsole, fontsize=\footnotesize]{haskell}
ghci> (run . runState 0 . runErr) decr
(Left (), 0)
  \end{minted}
  \pause
  \begin{minted}[bgcolor=bgConsole, fontsize=\footnotesize]{haskell}
ghci> (run . runErr . runState 0) decr
Left ()
  \end{minted}
\end{frame}
}

\begin{frame}
  \begin{columns}[c]
    \column{0.3\textwidth}
    \begin{center}
      \Large{\textcolor{RoyalBlue}{Monad Transformers}}
    \end{center}
    \column{0.3\textwidth}
    \column{0.3\textwidth}
    \begin{center}
      \Large{\textcolor{ForestGreen}{Algebraic Effect Handlers}}
    \end{center}
  \end{columns}
  \bigskip
  \bigskip
  \begin{columns}[c]
    \column{0.3\textwidth}
    \begin{center}
      \checkmark
    \end{center}
    \column{0.3\textwidth}
    \begin{center}
      \textbf{Effect Interaction}
    \end{center}
    \column{0.3\textwidth}
    \begin{center}
      \checkmark
    \end{center}
  \end{columns}
\end{frame}

%%%%%%%%%%%%%%%%%%%

\begin{thebibliography}{5}
  % \bibitem{Goldbach1742}[Goldbach, 1742]
  % Christian Goldbach.
  % \newblock A problem we should try to solve
\end{thebibliography}

%%%%%%%%%%%%%%%%%%%%%

\end{document}

