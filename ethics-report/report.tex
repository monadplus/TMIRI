
\documentclass[12pt, a4paper]{article}

\input{preamble}

\newcommand{\myhline}[0]{\par\noindent\rule{\textwidth}{0.4pt}\\}

\usepackage{fancyhdr}
\pagestyle{fancy}
\fancyhf{}
\rhead{Arnau Abella}
\lhead{TMIRI - UPC}
\rfoot{Page \thepage}

\title{%
  \vspace{-10ex}
  TMIRI: Ethics\\
  \Large{Robot priests: the rise of AI in religion}
}
\author{Arnau Abella}
\date{\today}

\begin{document}
\maketitle

\vspace{5ex}

The podcast titled \textit{``Robot priests: the rise of AI in religion''} raises several ethical dilemmas related to the use of \textit{AI} in the religious domain.

The first dilemma is related to \textbf{privacy and personal data}. Robotic priests are capable of recording people's private information during confessions and consultancies. This information could be misused by evil corporations or unintentionally leaked during the upload process. One possible way to prevent this malicious act would be having a stronger regulation on the usage of personal data by private and public corporations and institutions, and forbid the storage of this information in the persistent memory of the robot and prevent any kind of communication of the robot with external services during the span of these sessions.

The second dilemma that arises during the podcast is \textbf{ethical decisions taken by robots}. Believers will make decisions based on robotic priests' answers to their questions.
Since ancient times, humans have looked for answers in spiritual guides such as shamans or priest. The only difference, is that \textit{AIs}' responses are based on mathematical models, less or more complex, that lack what we people call ethics. Although, history have shown us that most religions have taken very questionable ethical decisions in the past such as the Christian crusades, the Inquisition or more contemporary affairs such as sexual harassment and pedophilia. To prevent this situation, \textit{AIs} should be trained with non-biased databases and models. And, decision should be taken by consensus on independent artificial intelligences and probably supervised by a group of human individuals.

The third subtle dilemma is the probable \textbf{worship of robots}.



The fourth dilemma that you will find on the last interview is the \textbf{lost of human identity}.


More dilemmas arises indirectly during the podcast.


\end{document}
