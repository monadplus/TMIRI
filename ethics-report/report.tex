
\documentclass[12pt, a4paper]{article}

\input{preamble}

\newcommand{\myhline}[0]{\par\noindent\rule{\textwidth}{0.4pt}\\}

\usepackage{fancyhdr}
\pagestyle{fancy}
\fancyhf{}
\lhead{Arnau Abella}
\rhead{TMIRI}
\rfoot{Page \thepage}

\title{%
  \vspace{-10ex}
  TMIRI: Ethics\\
  \Large{Robot priests: the rise of AI in religion}
}
\author{
  Arnau Abella\\
  \small{Universitat Polit\`ecnica de Catalunya}
}
\date{\today}

\begin{document}
\maketitle

\vspace{5ex}

The podcast titled \textit{``Robot priests: the rise of AI in religion''} raises several ethical dilemmas related to the use of \textit{AI} in the religious domain.

The first dilemma is related to \textbf{privacy and personal data}. Robotic priests are capable of recording people's private information during confessions and consultancies. This information could be misused by evil corporations or unintentionally leaked during the upload process. One possible way to prevent this malicious act would be having a stronger regulation on the usage of personal data by private and public corporations and institutions, and forbid the storage of this information in the persistent memory of the robot and prevent any kind of communication of the robot with external services during the span of these sessions.

The second dilemma that arises during the podcast is \textbf{ethical decisions taken by robots}. Believers will make decisions based on robotic priests' answers to their questions.
Since ancient times, humans have looked for answers in spiritual guides such as shamans or priest. The only difference, is that \textit{AIs}' responses are based on mathematical models, less or more complex, that lack what we people call ethics. Although, history have shown us that most religions have taken very questionable ethical decisions in the past such as the Christian crusades, the Inquisition or more contemporary affairs such as sexual harassment and pedophilia. To prevent this situation, \textit{AIs} should be trained with non-biased databases and models. And, decision should be taken by consensus on independent artificial intelligence and probably supervised by a group of human individuals.

\newpage

The third, more subtle, dilemma is the \textbf{worship of robots}. Robotic priests should be used as assistants during religious activities and should never replace the entity of the human priest. In less than a century, if research on \textit{AI} keeps the same growth as in 2020, computers will have similar computational power as the human brain. The more computational power the microprocessors have, the harder it will be to distinguish between a robot and a human. And, at some point in the future, computers will even surpass our intellectual abilities. In this theoretical future, artificial intelligence taking the role of world leaders is not so improbable. These leaders may be worship as superhumans or even gods! It may surprise you, however, human race have already started worshiping virtual entities around the world. This is happening in the entertainment industry in Asia. Countries like Japan, have a new modality of entertainment called \text{virtual youtubers}, also known as \textit{vtubers} which are viewed and followed by hundred of thousands of people. Vtubers are animated 3D models, operated by a human behind the scenes, that interact with other people and perform a wide range of virtual activities with them like chatting or playing videogames. It would be possible to replace, in short periods of time, the people behind these animated 3D models by top-level \textit{AIs} without anyone noticing it. From my point of view, a future where robots will have the same cognitive abilities as humans is inevitable. At that point in time, humanity should avoid repeating the errors of the past such as slavery, and treat both humans and robot with the same rights.

The fourth dilemma found in the last interview of the podcast is the \textbf{lost of human identity}. The more computational power the \textit{AIs} will have in the future, the more similar their reasoning will be to human beings. At some point, it will be almost impossible to distinguish between a human and an artificial intelligence. At this point, mankind will face an existential crisis because \textit{Homo Sapiens} will no longer be a unique specie. We will no longer have a characteristic trait that distinguishes us among the rest of the mammals \ldots we will be dispensable! I believe that one of the greatest sins of mankind is thinking that we are better than the rest of human beings. We torture, slaughter and mass murder other animals for our own selfish benefit to produce food, leathers or expensive decorations when there are alternative to all these needs.

To sum up, we have only discussed four ethical dilemmas related to the field \textit{Artificial Intelligence} but the number of dilemmas that arises in this field is unfathomable. At some point, \textit{humanity} will have to face these dilemmas and we should better prepare for that day.

\end{document}
