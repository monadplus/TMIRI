
\documentclass[12pt, a4paper]{article}

\input{preamble}

\newcommand{\myhline}[0]{\par\noindent\rule{\textwidth}{0.4pt}\\}

\usepackage{fancyhdr}
\pagestyle{fancy}
\fancyhf{}
\lhead{Arnau Abella}
\rhead{TMIRI}
\rfoot{Page \thepage}

\title{%
  \vspace{-10ex}
  The Social Dilemma\\
  \vspace{2ex}
  \large{Techniques and Methodology of Innovation \\and Research in Informatics}
}
\author{
  Arnau Abella\\
  \small{Universitat Polit\`ecnica de Catalunya}
}
\date{\today}

\begin{document}
\maketitle

\vspace{5ex}

The documentary \textit{"The Social Dilemma"} is related to the topic of ethics that we have been working on the \textit{Techniques and Methodology of Innovation and Research in Informatics}. This document will be structured in two parts: the first part will be a description of the documentary and outline of the most important topics. The second part will be focused on my personal thoughts on the topic and the role we have as software developers in the society.

The documentary focus on the topic of social media and exposes multiple implications of new technologies such as social networks and ML in the society. The exposition of the topics is done in a fluid and elegant way by mixing interviews with former directors and team leads of prestigious technology companies such as Facebook, Twitter, Pinterest, etcetera with a recreation by actors of the daily life of a fictional family that is suffering the negatve effects of unsupervised social media. The documentary is suitable for all audiences because the contents are explained in a superficial way to reach more public, its duration is about an hour and a half and it fulfills its objective in an entertaining way. The documentary is a call for the society to open its eyes and start taking action against the media manipulation of social media platforms such as Facebook, Youtube, Twitter, and many more. It is also a critique to big technology corporations that have no rules, no regulations and act as a sort of de facto government. I would recommend watching the documentary to anyone who uses these technologies, not to stop using them but to be aware of the massive power they have and how to prevent being manipulated by them.

After I finished watching \textit{"The Social Dilemma"} on Netflix, the recommendations engine, started reproducing one of the top 10 Netflix's tv shows of the month which was quite ironic. I would like to share my personal experience on the topic. My experience on the industry is very short but I have already faced some ethical dilemmas related to my work. I am specialized in statically typed functional languages such as Haskell or Agda and I have worked in several fields from web applications to decentralized cryptocurrencies. Those languages are not popular and the job offers are very specific. My last project was related to the creation of a new cryptocurrency. Working on a cryptocurrency based on the blockchain model arises some ethical concerns. I believe cryptocurrencies are not per se a negative technology but they tend to be used to speculate which wastes incredible amounts of energetic resources unless your proof of work is not computational power which is very rare. At that moment, I did not pay much attention to these dilemmas but I would do in the future. Another area where strongly statically typed languages are used such as Rust and Haskell is the creation of weapons and tools for the army that come with appealing salary offers but lots of moral implications.

As software engineers we are contributing to the software that models today's world and our choices will affect future generations. Each decision counts and ending in an utopia or dystopia is in our hands.

\end{document}
